\documentclass[11pt]{article}
\usepackage[top=2.1cm,bottom=2cm,left=2cm,right= 2cm]{geometry}
%\geometry{landscape}                % Activate for for rotated page geometry
\usepackage[parfill]{parskip}    % Activate to begin paragraphs with an empty line rather than an indent
\usepackage{graphicx}
\usepackage{amssymb}
\usepackage{epstopdf}
\usepackage{amsmath}
\usepackage{multirow}
\usepackage{hyperref}
\usepackage{changepage}
\usepackage{lscape}
\usepackage{ulem}
\usepackage{multicol}
\usepackage{dashrule}
\usepackage[usenames,dvipsnames]{color}
\usepackage{enumerate}
\newcommand{\urlwofont}[1]{\urlstyle{same}\url{#1}}
\newcommand{\degree}{\ensuremath{^\circ}}
\newcommand{\hl}[1]{\textbf{\underline{#1}}}



\DeclareGraphicsRule{.tif}{png}{.png}{`convert #1 `dirname #1`/`basename #1 .tif`.png}

\newenvironment{choices}{
\begin{enumerate}[(a)]
}{\end{enumerate}}

%\newcommand{\soln}[1]{\textcolor{MidnightBlue}{\textit{#1}}}	% delete #1 to get rid of solutions for handouts
\newcommand{\soln}[1]{ \vspace{1.35cm} }

%\newcommand{\solnMult}[1]{\textbf{\textcolor{MidnightBlue}{\textit{#1}}}}	% uncomment for solutions
\newcommand{\solnMult}[1]{ #1 }	% uncomment for handouts

%\newcommand{\pts}[1]{ \textbf{{\footnotesize \textcolor{black}{(#1)}}} }	% uncomment for handouts
\newcommand{\pts}[1]{ \textbf{{\footnotesize \textcolor{blue}{(#1)}}} }	% uncomment for handouts

\newcommand{\note}[1]{ \textbf{\textcolor{red}{[#1]}} }	% uncomment for handouts

\begin{document}


\enlargethispage{\baselineskip}

Spring 2021 \hfill Jingchen (Monika) Hu\\

\begin{center}
{\huge MATH 241 Homework 5}	\\
Due: Sunday 3/28 11:59pm to Moodle
\end{center}
\vspace{0.5cm}

\begin{itemize}

%%%%%%%%%%%%%%%%%%%%%%%%%%%%%%%%%%%%%%%%%%%%%%
  
    \item
    Chapter 4 Problem  35
    
    A box contains 5 red and 5 blue marbles. Two marbles are withdrawn randomly. If they are the same color, then you win \$1.10; if they are different colors, then you will -\$1.00. (That is, you lose \$1.00.) Calculate
    
    \begin{itemize}
    \item[(a)] the expected value of the amount you win;
    \item[(b)] the variance of the amount you win.
    \end{itemize}

 \item
    Chapter 4 Problem 40
    
    On a multiple-choice exam with 3 possible answers for each of the 5 questions, what is the probability that a student will get 4 or more correct answers just by guessing?
    
    
    \item
    Chapter 4 Problem 44
    
    A satellite system consists of $n$ components and functions on any given day if at least $k$ of the $n$ components function on that day. On a rainy day, each of the components independently functions with probability $p_1$, whereas on a dry day, each independently functions with probability $p_2$. If the probability of rain tomorrow is $\alpha$, what is the probability that the satellite system will function?
    

    \item
    Chapter 4 Problem  49
    
    When coin 1 is flipped, it lands on heads with probability 0.4; when coin 2 is flipped, it lands on heads with probability 0.7. One of these coins is randomly chosen and flipped 10 times.
    
    \begin{itemize}
    \item[(a)] What is the probability that the coin lands on heads on exactly 7 of the 10 flips?
    
    \item[(b)] Given that the first of these 10 flips lands heads, what is the conditional probability that exactly 7 of the 10 flips land on heads?
    \end{itemize}


    \item
    Chapter 4 Theoretical exercise 7
    
    Let $X$ be a random variable having expected value $\mu$ and variance $\sigma^2$. Find the expected value and variance of
    
    $$
    Y  = \frac{X - \mu}{\sigma}
    $$

 \item
    Chapter 4 Theoretical exercise 10. [Hint: use ${n + 1 \choose k + 1} = {n  \choose k } \frac{n + 1}{ k + 1}$]
    
    Let $X$ be a binomial random variable with parameters $n$ and $p$. Show that
    
    $$
    E\left[\frac{1}{X+1}\right] = \frac{1 - (1-p)^{n+1}}{(n+1)p}
    $$


\end{itemize}


\vspace{12pt}

\underline{Optional: if you feel like more practice}\\
These will not be graded, but you are welcome to discuss these with me during the office hour.

\begin{itemize}

%%%%%%%%%%%%%%%%%%%%%%%%%%%%%%%%%%%%%%%%%%%%%%

\item Textbook  Chapter 4 Problems:  37, 39, 41-42
\item Textbook  Chapter 4 Theoretical exercise: 8, 11, 13, 14

\end{itemize}








\end{document} 