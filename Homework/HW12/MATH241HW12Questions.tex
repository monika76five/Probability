\documentclass[11pt]{article}
\usepackage[top=2.1cm,bottom=2cm,left=2cm,right= 2cm]{geometry}
%\geometry{landscape}                % Activate for for rotated page geometry
\usepackage[parfill]{parskip}    % Activate to begin paragraphs with an empty line rather than an indent
\usepackage{graphicx}
\usepackage{amssymb}
\usepackage{epstopdf}
\usepackage{amsmath}
\usepackage{multirow}
\usepackage{hyperref}
\usepackage{changepage}
\usepackage{lscape}
\usepackage{ulem}
\usepackage{multicol}
\usepackage{dashrule}
\usepackage[usenames,dvipsnames]{color}
\usepackage{enumerate}
\newcommand{\urlwofont}[1]{\urlstyle{same}\url{#1}}
\newcommand{\degree}{\ensuremath{^\circ}}
\newcommand{\hl}[1]{\textbf{\underline{#1}}}



\DeclareGraphicsRule{.tif}{png}{.png}{`convert #1 `dirname #1`/`basename #1 .tif`.png}

\newenvironment{choices}{
\begin{enumerate}[(a)]
}{\end{enumerate}}

%\newcommand{\soln}[1]{\textcolor{MidnightBlue}{\textit{#1}}}	% delete #1 to get rid of solutions for handouts
\newcommand{\soln}[1]{ \vspace{1.35cm} }

%\newcommand{\solnMult}[1]{\textbf{\textcolor{MidnightBlue}{\textit{#1}}}}	% uncomment for solutions
\newcommand{\solnMult}[1]{ #1 }	% uncomment for handouts

%\newcommand{\pts}[1]{ \textbf{{\footnotesize \textcolor{black}{(#1)}}} }	% uncomment for handouts
\newcommand{\pts}[1]{ \textbf{{\footnotesize \textcolor{blue}{(#1)}}} }	% uncomment for handouts

\newcommand{\note}[1]{ \textbf{\textcolor{red}{[#1]}} }	% uncomment for handouts

\begin{document}


\enlargethispage{\baselineskip}

Spring 2021 \hfill Jingchen (Monika) Hu\\

\begin{center}
{\huge MATH 241 Homework 12}	\\
Due: Sunday 5/23 11:59pm to Moodle
\end{center}
\vspace{0.5cm}

\begin{itemize}

%%%%%%%%%%%%%%%%%%%%%%%%%%%%%%%%%%%%%%%%%%%%%%

    \item
    Chapter 7 Problem 75
    
    The moment generating function of $X$ is given by $M_X(t) = \exp\{2e^t - 2\}$ and that of $Y$ by $M_Y(t) = (\frac{3}{4}e^t + \frac{1}{4})^{10}$. If $X$ and $Y$ are independent, what are
    
    \begin{itemize}
    \item[(a)] $P\{X + Y = 2\}$?
    \item[(b)] $P\{XY = 0\}$?
    \item[(c)] $E[XY]$?
    \end{itemize}

    \item
    Chapter 7 Problem 77
    
    The joint density of $X$ and $Y$ is given by 
    
    $$
    f(x, y) = \frac{1}{\sqrt{2\pi}} e^{-y} e^{-(x-y)^2/2}, 0 < y < \infty, -\infty, x < \infty
    $$
    
    \begin{itemize}
    \item[(a)] Compute the joint moment generating function of $X$ and $Y$.
    \item[(b)] Compute the individual moment generating functions.
    \end{itemize}

    \item
    Chapter 7 Problem 79
    
    Successive weekly sales, in units of \$1,000, have a bivariate normal distribution with common mean 40, common standard deviation 6, and correlation 0.6.
    
    \begin{itemize}
    \item[(a)] Find the probability that the total of the next 2 weeks' sales exceeds 90.
    \item[(b)] If the correlation were 0.2 rather than 0.6, do you think that this would increase or decrease the answer to (a)? Explain your reasoning.
    \item[(c)] Repeat (a) when the correlation is 0.2.
    \end{itemize}

%    \item Theoretical exercise 13

%    \item Theoretical exercise 20
%
%    \item Theoretical exercise 29
%

      \item Chapter 8 Question 5
      
      Fifty numbers are rounded off to the nearest integer and then summed. If the individual round-off errors are uniformly distributed over (-0.5, 0.5), approximate the probability that the resultant sum differs from the exact sum by more than 3.
      
      \item Chapter 8 Question 6
      
      A die is continually rolled until the total sum of all rolls exceeds 300. Approximate the probability that at least 80 rolls are necessary.
\end{itemize}


\vspace{12pt}










\end{document} 