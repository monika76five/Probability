\documentclass[11pt]{article}
\usepackage[top=2.1cm,bottom=2cm,left=2cm,right= 2cm]{geometry}
%\geometry{landscape}                % Activate for for rotated page geometry
\usepackage[parfill]{parskip}    % Activate to begin paragraphs with an empty line rather than an indent
\usepackage{graphicx}
\usepackage{amssymb}
\usepackage{epstopdf}
\usepackage{amsmath}
\usepackage{multirow}
\usepackage{hyperref}
\usepackage{changepage}
\usepackage{lscape}
\usepackage{ulem}
\usepackage{multicol}
\usepackage{dashrule}
\usepackage[usenames,dvipsnames]{color}
\usepackage{enumerate}
\newcommand{\urlwofont}[1]{\urlstyle{same}\url{#1}}
\newcommand{\degree}{\ensuremath{^\circ}}
\newcommand{\hl}[1]{\textbf{\underline{#1}}}



\DeclareGraphicsRule{.tif}{png}{.png}{`convert #1 `dirname #1`/`basename #1 .tif`.png}

\newenvironment{choices}{
\begin{enumerate}[(a)]
}{\end{enumerate}}

%\newcommand{\soln}[1]{\textcolor{MidnightBlue}{\textit{#1}}}	% delete #1 to get rid of solutions for handouts
\newcommand{\soln}[1]{ \vspace{1.35cm} }

%\newcommand{\solnMult}[1]{\textbf{\textcolor{MidnightBlue}{\textit{#1}}}}	% uncomment for solutions
\newcommand{\solnMult}[1]{ #1 }	% uncomment for handouts

%\newcommand{\pts}[1]{ \textbf{{\footnotesize \textcolor{black}{(#1)}}} }	% uncomment for handouts
\newcommand{\pts}[1]{ \textbf{{\footnotesize \textcolor{blue}{(#1)}}} }	% uncomment for handouts

\newcommand{\note}[1]{ \textbf{\textcolor{red}{[#1]}} }	% uncomment for handouts

\begin{document}


\enlargethispage{\baselineskip}

Spring 2021 \hfill Jingchen (Monika) Hu\\

\begin{center}
{\huge MATH 241 Homework 1}	\\
Due: Sunday 2/28 11:59pm to Moodle
\end{center}
\vspace{0.5cm}

\textbf{Name:} \rule{6cm}{0.5pt}\\
\textbf{List the questions you hope to be explained in class} \rule{3cm}{0.5pt}	 \\


{\bf
\begin{itemize}
%\item Print out this cover page and staple with your homework.
\item Show all work. Incomplete solutions will be given no credit.
\item You may prepare either hand-written or typed solutions, but please make sure that they are legible.
Answers that cannot be read will be given no credit.
\item Textbook problems are numbered according to the 9th edition.
\item Read textbook Chapter 1 in the text before you start.

\end{itemize}
}

\underline{Textbook  Chapter 1   }

\begin{itemize}

%%%%%%%%%%%%%%%%%%%%%%%%%%%%%%%%%%%%%%%%%%%%%%

    \item
    Problem 3
    \item
    Problem 5

    \item
    Problem  7

    \item
    Problem  8

    \item
    Problem  13

    \item
    Problem  15

    \item
    Problem  19

    \item
    Problem  21

    \item
    Problem  22

    \item
    Problem  24

    \item
    Problem  27

    \item
    Theoretical exercise 2

    \item
    Theoretical exercise 8

    \item
    Theoretical exercise 9

    \item
    Theoretical exercise 13



\end{itemize}

\vspace{12pt}

\underline{Optional: if you feel like more practice}\\
These will not be graded, but you are welcome to discuss these with me during the office hour.

\begin{itemize}

%%%%%%%%%%%%%%%%%%%%%%%%%%%%%%%%%%%%%%%%%%%%%%

\item Textbook  Chapter 1 Problems: 1, 2, 4, 6, 9, 10, 11, 12, 16, 17, 18, 20, 23, 30
\item Textbook  Chapter 1 Theoretical exercise: 10, 11



\end{itemize}



\end{document} 