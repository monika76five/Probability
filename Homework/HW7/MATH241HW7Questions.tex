\documentclass[11pt]{article}
\usepackage[top=2.1cm,bottom=2cm,left=2cm,right= 2cm]{geometry}
%\geometry{landscape}                % Activate for for rotated page geometry
\usepackage[parfill]{parskip}    % Activate to begin paragraphs with an empty line rather than an indent
\usepackage{graphicx}
\usepackage{amssymb}
\usepackage{epstopdf}
\usepackage{amsmath}
\usepackage{multirow}
\usepackage{hyperref}
\usepackage{changepage}
\usepackage{lscape}
\usepackage{ulem}
\usepackage{multicol}
\usepackage{dashrule}
\usepackage[usenames,dvipsnames]{color}
\usepackage{enumerate}
\newcommand{\urlwofont}[1]{\urlstyle{same}\url{#1}}
\newcommand{\degree}{\ensuremath{^\circ}}
\newcommand{\hl}[1]{\textbf{\underline{#1}}}



\DeclareGraphicsRule{.tif}{png}{.png}{`convert #1 `dirname #1`/`basename #1 .tif`.png}

\newenvironment{choices}{
\begin{enumerate}[(a)]
}{\end{enumerate}}

%\newcommand{\soln}[1]{\textcolor{MidnightBlue}{\textit{#1}}}	% delete #1 to get rid of solutions for handouts
\newcommand{\soln}[1]{ \vspace{1.35cm} }

%\newcommand{\solnMult}[1]{\textbf{\textcolor{MidnightBlue}{\textit{#1}}}}	% uncomment for solutions
\newcommand{\solnMult}[1]{ #1 }	% uncomment for handouts

%\newcommand{\pts}[1]{ \textbf{{\footnotesize \textcolor{black}{(#1)}}} }	% uncomment for handouts
\newcommand{\pts}[1]{ \textbf{{\footnotesize \textcolor{blue}{(#1)}}} }	% uncomment for handouts

\newcommand{\note}[1]{ \textbf{\textcolor{red}{[#1]}} }	% uncomment for handouts

\begin{document}


\enlargethispage{\baselineskip}

Spring 2021 \hfill Monika (Jingchen) Hu\\

\begin{center}
{\huge MATH 241 Homework 7}	\\
Due: Sunday 4/18 11:59pm to Moodle
\end{center}
\vspace{0.5cm}


\begin{itemize}

%%%%%%%%%%%%%%%%%%%%%%%%%%%%%%%%%%%%%%%%%%%%%%
    \item
    Chapter 5 Problem 1
    
    Let $X$ be a random variable with probability density function
     
   \begin{align*}
   f(x) &= \begin{cases}
   c (1 - x^2)& \text{if $-1 \leq x \leq 1$} \\
   0 & \text{otherwise}
   \end{cases}
   \end{align*}
   
   \begin{itemize}
   \item[(a)] What is the value of $c$?
   \item[(b)] What is the cumulative distribution function of $X$?
   \end{itemize}
   
    \item
    Chapter 5 Problem 2
    
    A system consisting of one original unit plus a spare can function for a random amount of time $X$. If the density of $X$ is given (in units of months) by
    
   \begin{align*}
   f(x) &= \begin{cases}
   C x e^{-x/2} & x > 0\\
   0 & x \leq 0
   \end{cases}
   \end{align*}
   what is the probability that the system functions for at least 5 months?
   
    \item
    Chapter 5 Problem  6
    
    Compute $E[X]$ if $X$ has a density function given by
    
    \begin{itemize}
    \item[(a)] 
    
    \begin{align*}
   f(x) &= \begin{cases}
   \frac{1}{4}x e^{-x/2}& x > 0 \\
   0 & \text{otherwise}
   \end{cases}
   \end{align*}
   
   \item[(b)] 
   
   \begin{align*}
   f(x) &= \begin{cases}
   c (1 - x^2)& \text{if $-1 \leq x \leq 1$} \\
   0 & \text{otherwise}
   \end{cases}
   \end{align*}
   
   \item[(c)]
   
   \begin{align*}
   f(x) &= \begin{cases}
   \frac{5}{x^2} & x > 5 \\
   0 & x \leq 5
   \end{cases}
   \end{align*}
   
   
    \end{itemize}

    \item
    Chapter 5 Problem 10
    
    Trains headed for destination $A$ arrive at the train station at 15-minute intervals starting at 7 A.M., whereas trains headed for destination $B$ arrive at 15-minute intervals starting at 7:05 A.M.
    
    \begin{itemize}
    \item[(a)] If a certain passenger arrives at the station at a time uniformly distributed between 7 and 8 A.M. and then gets on the first train that arrives, what proportion of time does he or she go to destination $A$?
    
    \item[(b)] What if the passenger arrives at a time uniformly distributed between 7:10 and 8:10 A.M.?
    \end{itemize}
    

    \item
    Chapter 5 Problem 14
    
    Let $X$ be a uniform(0, 1) random variable. Compute $E[X^n]$ by using Proposition 2.1, and check the result by using the definition of expectation. 
    
    Proposition 2.1 is on textbook page 181: If $X$ is a continuous random variable with probability density function $f(x)$, then, for any real-valued function $g$, 
    $$
    E[g(X)] = \int_{-\infty}^{\infty}g(x)f(x)dx
    $$

   
    \item 
    Chapter 5 Theoretical exercise 2
    
    Show that
    $$
    E[Y] = \int_{0}^{\infty} P\{Y > y\} dy - \int_{0}^{\infty}P\{Y < -y\}dy
    $$

    Hint: Show that
    $$
    \int_0^{\infty}P\{Y < -y\}dy = - \int_{-\infty}^{0} x f_Y(x)dx
    $$
    
    $$
    \int_0^{\infty}P\{Y > y\} dy = \int_{0}^{\infty} x f_Y(x)dx
    $$
    
    \item 
    Chapter 5 Theoretical exercise 4
    
    Prove Corollary 2.1: $$E[aX+b] = aE[X] + b$$

    \item 
    Chapter 5 Theoretical exercise 7
    
    The standard deviation of $X$, denoted $SD(X)$, is given by
    $$
    SD(X) = \sqrt{Var(X)}
    $$
    Find $SD(aX+b)$ if $X$ has variance $\sigma^2$.
    
   

\end{itemize}


\vspace{12pt}

\underline{Optional: if you feel like more practice}\\
These will not be graded, but you are welcome to discuss these with me during the office hour.

\begin{itemize}

%%%%%%%%%%%%%%%%%%%%%%%%%%%%%%%%%%%%%%%%%%%%%%

\item Textbook  Chapter 5 Problems: 3-5, 7-8, 11-13
\item Textbook  Chapter 5 Theoretical exercise: 1

\end{itemize}








\end{document} 