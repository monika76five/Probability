\documentclass[11pt]{article}
\usepackage[top=2.1cm,bottom=2cm,left=2cm,right= 2cm]{geometry}
%\geometry{landscape}                % Activate for for rotated page geometry
\usepackage[parfill]{parskip}    % Activate to begin paragraphs with an empty line rather than an indent
\usepackage{graphicx}
\usepackage{amssymb}
\usepackage{epstopdf}
\usepackage{amsmath}
\usepackage{multirow}
\usepackage{hyperref}
\usepackage{changepage}
\usepackage{lscape}
\usepackage{ulem}
\usepackage{multicol}
\usepackage{dashrule}
\usepackage[usenames,dvipsnames]{color}
\usepackage{enumerate}
\newcommand{\urlwofont}[1]{\urlstyle{same}\url{#1}}
\newcommand{\degree}{\ensuremath{^\circ}}
\newcommand{\hl}[1]{\textbf{\underline{#1}}}



\DeclareGraphicsRule{.tif}{png}{.png}{`convert #1 `dirname #1`/`basename #1 .tif`.png}

\newenvironment{choices}{
\begin{enumerate}[(a)]
}{\end{enumerate}}

%\newcommand{\soln}[1]{\textcolor{MidnightBlue}{\textit{#1}}}	% delete #1 to get rid of solutions for handouts
\newcommand{\soln}[1]{ \vspace{1.35cm} }

%\newcommand{\solnMult}[1]{\textbf{\textcolor{MidnightBlue}{\textit{#1}}}}	% uncomment for solutions
\newcommand{\solnMult}[1]{ #1 }	% uncomment for handouts

%\newcommand{\pts}[1]{ \textbf{{\footnotesize \textcolor{black}{(#1)}}} }	% uncomment for handouts
\newcommand{\pts}[1]{ \textbf{{\footnotesize \textcolor{blue}{(#1)}}} }	% uncomment for handouts

\newcommand{\note}[1]{ \textbf{\textcolor{red}{[#1]}} }	% uncomment for handouts

\begin{document}


\enlargethispage{\baselineskip}

Spring 2021 \hfill Monika (Jingchen) Hu\\

\begin{center}
{\huge MATH 241 Homework 8}	\\
Due: Sunday 4/25 11:59pm to Moodle
\end{center}
\vspace{0.5cm}


\begin{itemize}

%%%%%%%%%%%%%%%%%%%%%%%%%%%%%%%%%%%%%%%%%%%%%%

    \item
    Chapter 5 Problem 15
    
    If $X$ is a normal random variable with parameters $\mu = 10$ and $\sigma^2 = 36$, compute
    
    \begin{itemize}
    \item[(a)] $P\{X > 5\}$;
    \item[(b)] $P\{4 < X < 16\}$;
    \item[(c)] $P\{X < 8\}$;
    \item[(d)] $P\{X < 20\}$;
    \item[(e)] $P\{X > 16\}$.
    \end{itemize}
    
    \item
    Chapter 5 Problem 16
    
    The annual rainfall (in inches) in a certain region is normally distributed with $\mu = 40$ and $\sigma = 4$. What is the probability that starting with this year, it will take more than 10 years before a year occurs having a rainfall of more than 50 inches? What assumptions are you making?

    \item
    Chapter 5 Problem  18
    
    Suppose that $X$ is a normal random variable with mean 5. If $P\{X > 9\} = 0.2$, approximately what is $Var(X)$? 
    
    \item 
    Chapter 5 Problem 20
    
    If 65 percent of the population of a large community is in favor of a proposed rise in school taxes, approximate the probability that a random sample of 100 people will contain
    
    \begin{itemize}
    \item[(a)] at least 50 who are in favor of the proposition;
    \item[(b)] between 60 and 70 inclusive who are in favor;
    \item[(c)] fewer than 75 in favor.
    \end{itemize}
    
    \item
    Chapter 5 Problem 23
    
    One thousand independent rolls of a fair die will be made. Compute an approximation to the probability that the number 6 will appear between 150 and 200 times inclusively. If the number 6 appears exactly 200 times, find the probability that the number 5 will appear less than 150 times.
    
    \item 
    Chapter 5 Problem 28
    
    Twelve percent of the population is left handed. Approximate the probability that there are at least 20 left-handers in a school fo 200 students. State your assumptions.
    
    \item 
    Chapter 5 Problem 37
    
    If $X$ is uniformly distributed over (-1, 1), find
    
    \begin{itemize}
    \item[(a)] $P\{|X| > \frac{1}{2}\}$;
    \item[(b)] the density function of the random variable $|X|$.
    \end{itemize}
    
    \item 
    Chapter 5 Problem 39
    
    If $X$ is an exponential random variable with parameter $\lambda = 1$, compute the probability density function of the random variable $Y$ defined by $Y = \log X$. Note that the pdf of an exponential random variable with rate $\lambda > 0$ is: $f(x) = \lambda e^{-\lambda x}, x > 0$.
    
    \item 
    Chapter 5 Problem 40
    
    If $X$ is uniformly distributed over (0, 1), find the density function of $Y = e^X$.
    

    \item 
    Chapter 5 Theoretical exercise 13 (only part a and b)
    
    The median of a continuous random variable having distribution function $F$ is that value $m$ such that $F(m) = \frac{1}{2}$. That is, a random variable is just as likely to be larger than its median as it is to be smaller. Find the median of $X$ if $X$ is
    
    \begin{itemize}
    \item[(a)] uniformly distributed over (a, b);
    \item[(b)] normal with parameters $\mu, \sigma^2$;
    \end{itemize}
    
    \item 
    
    Chapter 5 Theoretical exercise 29
    
    Let $X$ be a continuous random variable having cumulative distribution function $F$. Define the random variable $Y$ by $Y = F(X)$. Show that $Y$ is uniformly distributed over (0, 1).
    
    



\end{itemize}


\vspace{12pt}

\underline{Optional: if you feel like more practice}\\
These will not be graded, but you are welcome to discuss these with me during the office hour.

\begin{itemize}

%%%%%%%%%%%%%%%%%%%%%%%%%%%%%%%%%%%%%%%%%%%%%%

\item Textbook  Chapter 5 Problems: 17, 19, 21, 24-27, 29-30, 22, 41-42
\item Textbook  Chapter 5 Theoretical exercise: 9, 10, 12, 14, 15, 18, 27, 30-31

\end{itemize}








\end{document} 