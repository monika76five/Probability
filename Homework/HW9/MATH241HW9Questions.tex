\documentclass[11pt]{article}
\usepackage[top=2.1cm,bottom=2cm,left=2cm,right= 2cm]{geometry}
%\geometry{landscape}                % Activate for for rotated page geometry
\usepackage[parfill]{parskip}    % Activate to begin paragraphs with an empty line rather than an indent
\usepackage{graphicx}
\usepackage{amssymb}
\usepackage{epstopdf}
\usepackage{amsmath}
\usepackage{multirow}
\usepackage{hyperref}
\usepackage{changepage}
\usepackage{lscape}
\usepackage{ulem}
\usepackage{multicol}
\usepackage{dashrule}
\usepackage[usenames,dvipsnames]{color}
\usepackage{enumerate}
\newcommand{\urlwofont}[1]{\urlstyle{same}\url{#1}}
\newcommand{\degree}{\ensuremath{^\circ}}
\newcommand{\hl}[1]{\textbf{\underline{#1}}}



\DeclareGraphicsRule{.tif}{png}{.png}{`convert #1 `dirname #1`/`basename #1 .tif`.png}

\newenvironment{choices}{
\begin{enumerate}[(a)]
}{\end{enumerate}}

%\newcommand{\soln}[1]{\textcolor{MidnightBlue}{\textit{#1}}}	% delete #1 to get rid of solutions for handouts
\newcommand{\soln}[1]{ \vspace{1.35cm} }

%\newcommand{\solnMult}[1]{\textbf{\textcolor{MidnightBlue}{\textit{#1}}}}	% uncomment for solutions
\newcommand{\solnMult}[1]{ #1 }	% uncomment for handouts

%\newcommand{\pts}[1]{ \textbf{{\footnotesize \textcolor{black}{(#1)}}} }	% uncomment for handouts
\newcommand{\pts}[1]{ \textbf{{\footnotesize \textcolor{blue}{(#1)}}} }	% uncomment for handouts

\newcommand{\note}[1]{ \textbf{\textcolor{red}{[#1]}} }	% uncomment for handouts

\begin{document}


\enlargethispage{\baselineskip}

Spring 2021 \hfill Jingchen (Monika) Hu\\

\begin{center}
{\huge MATH 241 Homework 9}	\\
Due: Sunday 5/2 11:59pm to Moodle
\end{center}
\vspace{0.5cm}


\begin{itemize}

%%%%%%%%%%%%%%%%%%%%%%%%%%%%%%%%%%%%%%%%%%%%%%
    \item
    Chapter 6 Problem  2
    
    Suppose that 3 balls are chosen without replacement from an urn consisting of 5 white and 8 red balls. Let $X_i$ equal 1 if the $i$th ball selected is white, and let it equal 0 otherwise. Give the joint probability mass function of
    
    \begin{itemize}
    \item[(a)] $X_1, X_2$;
    \item[(b)] $X_1, X_2, X_3$.
    \end{itemize}

    \item
    Chapter 6 Problem  7
    
    Consider a sequence of independent Bernoulli trials, each of which is a success with probability $p$. Let $X_1$ be the number of failures preceding the first success, and let $X_2$ be the number of failures between the first two successes. Find the joint mass function of $X_1$ and $X_2$.

    \item
    Chapter 6 Problem  8
    
    The joint probability density function of $X$ and $Y$ is given by
    
    $$
    f(x, y) = c(y^2 - x^2)e^{-y}, -y \leq x \leq y, 0 < y < \infty
    $$
    
    \begin{itemize}
    \item[(a)] Find $c$.
    \item[(b)] Find the marginal densities of $X$ and $Y$.
    \item[(c)] Find $E[X]$.
    \end{itemize}

    \item
    Chapter 6 Problem  9
    
    The joint probability density function of $X$ and $Y$ is given by
    
    $$
    f(x, y) = \frac{6}{7}(x^2 + \frac{xy}{2}), 0 < x < 1, 0 < y < 2
    $$
    
    \begin{itemize}
    \item[(a)] Verify that this is indeed a joint density function.
    \item[(b)] Compute the density function of $X$.
    \item[(c)] Find $P\{X > Y\}$.
    \item[(d)] Find $P\{Y > \frac{1}{2} \mid X < \frac{1}{2}\}$.
    \item[(e)] Find $E[X]$.
    \item[(f)] Find $E[Y]$.
    \end{itemize}
    

    \item
    Chapter 6 Problem  15
    
    The random vector $(X, Y)$ is said to be uniformly distributed over a region $R$ in the plane if, for some constant $c$, its joint density is
    
    \begin{equation*}
    f(x, y) =
    \begin{cases}
      c, & \text{if $(x, y) \in R$} \\
      0, & \text{otherwise}
    \end{cases}
  \end{equation*}
  
    \begin{itemize}
    \item[(a)] Show that $1/c$ = area of region $R$.
    \end{itemize}
    
    Suppose that $(X, Y)$ is uniformly distributed over the square centered at (0, 0) and with sides of length 2.
    
    \begin{itemize}
    \item[(b)] Show that $X$ and $Y$ are independent, with each being distributed uniformly over (-1, 1).
    \item[(c)] What is the probability that $(X, Y)$ lies int he circle of radius 1 centered at the origin? That is, find $P\{X^2 + Y^2 \leq 1\}$.
    \end{itemize}

    \item
    Chapter 6 Problem 18
    
    Two points are selected randomly on a line of length $L$ so as to be on opposite sides of the midpoint of the line. [In other words, the two points $X$ and $Y$ are independent random variables such that $X$ is uniformly distributed over (0, $L/2$) and $Y$ is uniformly distributed over ($L/2, L$).] Find the probability that the distance between the two points is greater than $L/3$.

    \item
    Chapter 6 Problem 20
    
    The joint density of $X$ and $Y$ is given by
    
    \begin{equation*}
    f(x, y) =
    \begin{cases}
      xe^{-(x+y)}, & x > 0, y > 0 \\
      0, & \text{otherwise}
    \end{cases}
    \end{equation*}
  
    Are $X$ and $Y$ independent? If, instead, $f(x, y)$ were given by
    
    \begin{equation*}
    f(x, y) =
    \begin{cases}
      2, & 0 < x < y, 0 < y < 1 \\
      0, & \text{otherwise}
    \end{cases}
    \end{equation*}
    
    would $X$ and $Y$ be independent?

\end{itemize}


\vspace{12pt}

\underline{Optional: if you feel like more practice}\\
These will not be graded, but you are welcome to discuss these with me during the office hour.

\begin{itemize}

%%%%%%%%%%%%%%%%%%%%%%%%%%%%%%%%%%%%%%%%%%%%%%

\item Textbook  Chapter 6 Problems: 1, 4-6, 10-14, 17, 19, 21-23
%\item Textbook  Chapter 5 Theoretical exercise: 12, 14, 15, 18, 27, 30-31

\end{itemize}








\end{document} 