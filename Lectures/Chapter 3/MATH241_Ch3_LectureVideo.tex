\documentclass[slidestop,compress,mathserif]{beamer}
%\documentclass[slidestop,compress,mathserif,handout]{beamer}

%\documentclass[xcolor=dvipsnames,handout]{beamer}
%\documentclass[xcolor=dvipsnames]{beamer}

%\documentclass[handout]{beamer}

%%% To get rid of solutions on handouts:
\newcommand{\soln}[1]{\textit{\textcolor{darkGray}{#1}}}				% For slides
%\newcommand{\soln}[1]{ }	% For handouts

% to get pausing to work properly on slides
\newcommand{\hide}[1]{#1}	% For slides
%\newcommand{\hide}[1]{ }	% For handouts


%\usepackage{multicol}
\usepackage{amsfonts}
%\usepackage[pdftex,dvipsnames]{color}
\usepackage{graphicx}
\usepackage{subfigure}
%\usepackage{picinpar}
\usepackage{pifont}
\usepackage{pgf,pgfarrows,pgfnodes}
%\usepackage{wasysym,manfnt,phaistos,empheq}
\usepackage[english]{babel}
\usepackage{pgfpages}
\usepackage{natbib}
\usepackage{hyperref}
\usepackage{multimedia}
%\usepackage{amsfonts,amstext,amssymb,amsbsy,amsopn,amsthm,eucal,latexsym,mathrsfs}
\usepackage{amsmath,amsfonts,amstext,amssymb,amsbsy,amsopn,amsthm,eucal,latexsym,mathrsfs}
\usepackage{ulem}
\usepackage{setspace}
\usepackage{array}
%\usepackage{rotating}
\usepackage{multirow}
\usepackage{verbatim}
\usepackage{multicol}

\setbeamertemplate{navigation symbols}{}

%\usepackage{tikz}
%\usetikzlibrary{arrows,shapes,trees,backgrounds}


%\setbeameroption{show notes on second screen}
%\setbeameroption{show notes}
%\setbeameroption{show only notes}

\definecolor{links}{HTML}{2A1B81}
\hypersetup{colorlinks,linkcolor=,urlcolor=links}

\newtheorem*{principle}{Inscrutibility Principle}
\newtheorem*{punchline}{Punch Line}
\newtheorem{defn}{Definition}

\definecolor{Scarlet}{RGB}{140,17,17}
\definecolor{VassarRed}{RGB}{128,0,0}

% "dinglist" environment
  \renewenvironment{dinglist}[2][blue]
  {\begin{list}{\textcolor{blue}{\ding{#2}}}{}}{\end{list}}
  % Symbol definitions for these lists
  \newcommand{\DingListSymbolA}{43}
  \newcommand{\DingListSymbolB}{243}
  \newcommand{\DingListSymbolC}{224}
  \newcommand{\DingListSymbolD}{219}
  \newcommand{\DingListSymbolCheck}{52}
  \newcommand{\DingListSymbolCross}{56}


  \newenvironment{ballotenv}
{\only{%
\setbeamertemplate{itemize item}{\ding{45}}%
\setbeamertemplate{itemize subitem}{\ding{46}}%
\setbeamertemplate{itemize subsubitem}{\ding{46}}}} {}
\setbeamertemplate{itemize item}{\ding{49}}
\setbeamertemplate{itemize subitem}{\ding{47}}
\setbeamertemplate{itemize subsubitem}{\ding{47}}


%User defined colors: See colors section
\xdefinecolor{oiBlue}{rgb}{0.15, 0.35, 0.55}
\xdefinecolor{gray}{rgb}{0.5, 0.5, 0.5}
\xdefinecolor{darkGray}{rgb}{0.3, 0.3, 0.3}
\xdefinecolor{darkerGray}{rgb}{0.2, 0.2, 0.2}
\xdefinecolor{rubineRed}{rgb}{0.89,0,0.30}
\xdefinecolor{linkCol}{rgb}{0.11,0.49,0.95}	
\xdefinecolor{irishGreen}{rgb}{0,0.60,0}	
\xdefinecolor{darkturquoise}{rgb}{0.44, 0.58, 0.86}
\definecolor{lightGreen}{rgb}{0.533,0.765,0.42}
\xdefinecolor{Regalia}{HTML}{522D80}
\xdefinecolor{ClemsonOrange}{HTML}{EA6A20}

\definecolor{duke@LightGrey}{RGB}{200,200,200}\definecolor{DarkGreen}{RGB}{0,100,0}
\definecolor{Oranges}{RGB}{255,127,0}
\definecolor{LightGray}{RGB}{211,211,211}

%\setbeamertemplate{footline}{%
%  \raisebox{5pt}{\makebox[\paperwidth]{\hfill\makebox[10pt]{\scriptsize\insertframenumber}}}}

\setbeamercolor{equation background}{fg=black,bg=duke@LightGrey}
  % Boxed equation
  \newcommand{\eqbox}[2][0.6]{%
  \centerline{
  \begin{beamerboxesrounded}[lower=equation background,width=#1\hsize,shadow=true]{}
\parbox{#1\hsize}{%
      \[
        \textcolor{black} {#2}
      \]}
  \end{beamerboxesrounded}
}}

\AtBeginSection[] {
  \begin{frame}<beamer>\frametitle{Outline}
    \tableofcontents[currentsection,hideothersubsections]
  \end{frame}
}
%
%
%\AtBeginSubsection[] {
%  \begin{frame}<beamer>\frametitle{Outline}
%    \tableofcontents%[currentsection,currentsubsection]
%  \end{frame}
%}

%\usecolortheme[RGB={82,45,128}]{structure}
%\usecolortheme[RGB={162,80,22}]{structure}
\usecolortheme[RGB={128,0,0}]{structure}
\usetheme[secheader]{Boadilla}
%\usetheme[height=7mm]{Rochester}
%\usetheme{Copenhagen}
%\usetheme{Antibes}
%\usecolortheme{seahorse}
%\usecolortheme{crane}
%\usecolortheme{rose}
%\usefonttheme[onlylarge]{structurebold}
%\usefonttheme[onlymath]{serif}



\def\diag{{\rm diag}}


\def\E{\mathbb{E}}
\def\Prob{\mathbb{P}}
\def\argmin{{\rm argmin}}
\def\argmax{{\rm argmax}}
\def\Def{\stackrel{def}{=}}


\newtheorem{assumption}{Assumptions}
\newtheorem*{proposition}{Proposition}
\newtheorem*{remark}{Remark}



%\setbeamercolor{disc title}{bg=oiBlue!40!white!60,fg=blue}
\setbeamercolor{disc body}{bg= Regalia!20!white!80,fg= Regalia!80!black!90}

\setbeamercolor{clicker ungraded title}{bg=irishGreen!80!white!50,fg=irishGreen!30!black!90}
\setbeamercolor{clicker ungraded body}{bg=irishGreen!20!white!80,fg=irishGreen!30!black!90}

\setbeamercolor{clicker review title}{bg=gray!80!white!80,fg=oiBlue!80!black!90}
\setbeamercolor{clicker review body}{bg=gray!30!white!90,fg=oiBlue!80!black!90}

\setbeamercolor{code body}{bg=gray!20!white!80,fg=black}


% Custom commands
\newcommand{\degree}{\ensuremath{^\circ}}
\newcommand{\Note}[1]{
\rule{2.5cm}{0.25pt} \\ \textit{\scriptsize {\textcolor{rubineRed}{Note:} \textcolor{gray}{#1}}}}
\newcommand{\ct}[1]{
\vfill
{\tiny #1}}
\newcommand{\Remember}[1]{\textit{\scriptsize{\textcolor{orange}{Remember:} \textcolor{gray}{#1}}}}
\newcommand{\red}[1]{\textit{\textcolor{rubineRed}{#1}}}
\newcommand{\pink}[1]{\textit{\textcolor{rubineRed!90!white!50}{#1}}}
\newcommand{\green}[1]{\textit{\textcolor{irishGreen}{#1}}}
\newcommand{\webURL}[1]{\urlstyle{same}\textit{\textcolor{linkCol}{\url{#1}}} }
\newcommand{\webLink}[2]{\href{#1}{\textcolor{linkCol}{{#2}}}}
\newcommand{\mail}[1]{\href{mailto:#1}{\textit{\textcolor{linkCol}{#1}}}}
\newcommand{\hl}[1]{\textit{\textcolor{oiBlue}{#1}}}
\newcommand{\hlGr}[1]{\textit{\textcolor{lightGreen}{#1}}}
\newcommand{\mathhl}[1]{\textcolor{oiBlue}{\ensuremath{#1}}}
\newcommand{\ex}[1]{\textcolor{blue}{{{\small (#1)}}}}
\newcommand{\disc}[1]{
\begin{beamerboxesrounded}[shadow = true, lower = disc body, upper = disc title]{}
#1
\end{beamerboxesrounded}
}

\newcommand{\cl}[1]{
\begin{beamerboxesrounded}[shadow = true, lower = clicker ungraded body, upper = clicker ungraded title]{Question}
$\:$ \\
#1
\end{beamerboxesrounded}
}

\newcommand{\clR}[1]{
\begin{beamerboxesrounded}[shadow = true, lower = clicker review body, upper = clicker review title]{\red{Review question} }
$\:$ \\
#1
\end{beamerboxesrounded}
}

\newcommand{\formula}[2]{
\begin{beamerboxesrounded}[shadow = true, lower = white, upper = clicker review body]{#1}
#2
\end{beamerboxesrounded}
$\:$ \\
}

\newenvironment{twocol}[4]{
\begin{columns}[c]
\column{#1\textwidth}
#3
\column{#2\textwidth}
#4
\end{columns}
}


\newenvironment{slot}[2]{
\begin{array}{c}
\underline{#1} \\
#2
\end{array}
}

\newcommand{\pr}[1]{
\left( #1 \right)
}

\newcommand{\solnMult}[1]{
\item[] \vspace{-0.59cm}
\only<beamer| beamer:1>{\item #1}
\soln{\only<2->{\item \red{#1}}}
}

%\newcommand{\codechunk}[1]{
%\begin{beamerboxesrounded}[shadow = true, lower = code body]{}
%{\small #1}
%\end{beamerboxesrounded}
%}

% Change margin

\newenvironment{changemargin}[2]{%
\begin{list}{}{%
\setlength{\topsep}{0pt}%
\setlength{\leftmargin}{#1}%
\setlength{\rightmargin}{#2}%
\setlength{\listparindent}{\parindent}%
\setlength{\itemindent}{\parindent}%
\setlength{\parsep}{\parskip}%
}%
\item[]}{\end{list}}

% Footnote

\long\def\symbolfootnote[#1]#2{\begingroup%
\def\thefootnote{\fnsymbol{footnote}}\footnote[#1]{#2}\endgroup}

% Commands from the book
\newenvironment{data}[1]{\texttt{#1}}{}
\newenvironment{var}[1]{\texttt{#1}}{}
\newenvironment{resp}[1]{\texttt{#1}}{}






%%%%%%%%%%%%%%%%%%%%%%%%%%%%%%%%%%%%%%%%%%%%%%%%%%%%%%%%%%%%%%%%%%%%%%%%%%%%%%%%%%%%%%%%%%%%%%%

\title[Chapter 3]{Chapter 3}
\subtitle{Conditional Probability and Independence}

%%%%%%%%%%%%%%%%%%%%%%%%%%%%%%%%%%%%%%%%%%%%%%%%%%%%%%%%%%%%%%%%%%%%%%%%%%%%%%%%%%%%%%%%%%%%%%%


\author[Jingchen (Monika) Hu] % (optional, use only with lots of authors)
{Jingchen (Monika) Hu}
% - Give the names in the same order as the appear in the paper.
% - Use the \inst{?} command only if the authors have different
%   affiliation.

\institute[Vassar] % (optional, but mostly needed)
{Vassar College}
% - Use the \inst command only if there are several affiliations.
% - Keep it simple, no one is interested in your street address.

\date[MATH 241] % (optional, should be abbreviation of conference name)
{MATH 241}
% - Either use conference name or its abbreviation.
% - Not really informative to the audience, more for people (including
%   yourself) who are reading the slides online

\subject{MATH 241}
% This is only inserted into the PDF information catalog. Can be left
% out.



% If you wish to uncover everything in a step-wise fashion, uncomment
% the following command:

%\beamerdefaultoverlayspecification{<+->}


\begin{document}



%%%%%%%%%%%%%%%%%%%%%

% Title Page

\begin{frame}%[plain]
\titlepage
\end{frame}

%%%%%%%%%%%%%%%%%%%%%%
%\addtocounter{framenumber}{-1}
%
%\begin{frame}\frametitle{Announcement and comments}
%
%\begin{itemize}
%\item Quiz 1: extra credit appointment (TWTh this week, M next week)\pause
%\item Problems from HW1: TE2, TE8, and TE13
%\end{itemize}
%
%\end{frame}
%%





%%%%%%%%%%%%%%%%%%%%%%%%%%%%%%%%%%%%%%%%%%
\section{Conditional probability}
%%%%%%%%%%%%%%%%%%%%%%%%%%%%%%%%%%%%%%%%%%
\begin{frame}\frametitle{Conditional probability: motivation}

The probability of getting a one when rolling a fair 6-sided die is 1/6\\
\vspace{0.5cm}\pause
Suppose you were given the extra information that the die roll was an odd number (hence 1, 3 or 5)\\
\vspace{0.5cm}\pause

\red{Conditional on this new information}, the probability of a one is now 1/3


\end{frame}

%%%%%%%%%%%%%%%%%%%%%%%%%%%%%%%%%%%%%%%%%%
\begin{frame}\frametitle{Conditional probability}

\begin{defn}
Given two events $A$ and $B$ with $P(B) > 0$, the \hl{conditional
probability} of $A$ given $B$ has occurred is defined as
\[P(A\mid B)=\frac{P(A\cap B)}{P(B)}\]
\end{defn}
\pause
\begin{itemize}
\item When $B$ is the sample space: $P(A \mid B) = P(A)$
\item Intuition: in a sample space with equally likely outcomes,
\[P(A \mid B) = \frac{\#(A \cap B)}{\#(B)}\]
\end{itemize}

\end{frame}


%%%%%%%%%%%%%%%%%%%%%%%%%%%%%%%%%%%%%%%%%%
\begin{frame}\frametitle{Example}
Consider the die roll example:  $B = \{1~\text{or}~3~\text{or}~5\}, A=\{1\}$
\begin{align*}
P(\text{get 1 given that roll is odd}) =& P(A \mid B)\\
= & \frac{P(A\cap B)}{P(B)} \\
=& \frac{P(\text{get 1})}{P(\text{get 1 or 3 or 5})} \\
=& \frac{1/6}{3/6}=\frac{1}{3}
\end{align*}
\end{frame}


%%%%%%%%%%%%%%%%%%%%%%%%%%%%%%%%%%%%%%%%%%
\begin{frame}
\disc{Example: roll two fair 6-sided dice.
Find the probability of the sum of two rolls is 7, given the condition that the first roll is 4.}
\pause
Denote two events $A=\{\text{Sum is 7}\}$ and $B=\{\text{First roll is 4}\}$.\\
Find $P(A \mid B)$.
Formulas to use?
\[P(A\mid B)=\frac{P(A\cap B)}{P(B)}\]

\pause In context,
\[
P(A \cap B) = P\left(\{(4, 3)\}\right) = 1/36, \quad P(B) = 1/6
\]
\[
P(A \mid B) = \frac{1/36}{1/6} = \frac{1}{6}
\]

\end{frame}



%%%%%%%%%%%%%%%%%%%%%%%%%%%%%%%%%%%%%%%%%%
\begin{frame}\frametitle{Propositions of conditional probability}

\begin{enumerate}
\item $P(A \mid A) = 1$ \pause
\vspace{1.5cm}
\item $P(A^c \mid A) = 0$ \pause
\vspace{1.5cm}
\item $P(A^c \mid B) = 1 - P(A \mid B)$
\vspace{1.5cm}
\end{enumerate}

%{\it proof?}

\end{frame}

%%%%%%%%%%%%%%%%%%%%%%%%%%%%%%%%%%%%%%%%%%
\begin{frame}\frametitle{Multiplication rule}


\begin{dinglist}{\DingListSymbolA}
\item By the definition of conditional probability, the \hl{joint probability} of $A$ and $B$ is
\[ P(A \cap B) = P(A \mid B)P(B) \]
  \begin{itemize}
  \item Usually, $P(A)$ and $P(B)$ are called \hl{marginal probabilities}.
  \end{itemize}

\vspace{0.5cm}
\pause
\item Generalize to $n$ events: chaining of probabilities
\end{dinglist}
%\footnotesize{
\begin{align*}
P\left(\bigcap_{i=1}^n A_i\right)
%	& = P(A_1) \cdot \frac{P(A_2 A_1)}{P(A_1)} \cdot \frac{P(A_3   A_2   A_1)}{P(A_2   A_1)}\cdots \frac{P(A_1 \ldots A_n)}{P(A_1 \ldots A_{n-1})}\\
	& = P(A_1) P(A_2 \mid A_1) P(A_3 \mid  A_1, A_2) \cdots P(A_n  \mid  A_1,\ldots,A_{n-1})\\
\end{align*}
%}

%{\it We've already used it even before officially introduced...}


\end{frame}


%%%%%%%%%%%%%%%%%%%%%%%%%%%%%%%%%%%%%%%%%%
\begin{frame}%\frametitle{Multiplication rule}
\disc{A box contains 8 white balls and 4 black ones. 3 balls are drawn at random without replacement.
\begin{enumerate}
\item What's the probability of getting $W_1B_2B_3$?
\item What's the probability of getting black in draw 2?
\end{enumerate}
}

\pause
\begin{enumerate}
\item Multiplication rule
\begin{align*}
P(W_1B_2B_3) & = P(W_1) P(B_2 \mid W_1) P(B_3 \mid W_1B_2)\\
			& = \frac{8}{12}\cdot \frac{4}{11} \cdot \frac{3}{10}
\end{align*}
\vspace{-0.3cm}
\pause
\item If we want to consider draw 1, which can be $W_1$ or $B_1$ (disjoint!)
\begin{align*}
P(B_2) & = P(W_1B_2) + P(B_1B_2)\\
			& = \frac{8}{12}\cdot \frac{4}{11} + \frac{4}{12} \cdot \frac{3}{11}
			 = \frac{44}{132} = \frac{4}{12}
\end{align*}
%\pause
%Actually we do not need to consider draw 1. The marginal probability
%\[P(B_2) = \frac{4}{12}\]

\end{enumerate}
\end{frame}



%%%%%%%%%%%%%%%%%%%%%%%%%%%%%%%%%%%%
\begin{frame}\frametitle{Law of total probability}

For events $A_1,\ldots,A_n$  are {\bf disjoint}, and
\[ \bigcup_{i=1}^n A_i = S,\]
then for any event $B$, \pause
\begin{dinglist}{\DingListSymbolA}
\item  law of total probability
\[P(B)=P(B \mid A_1)P(A_1)+\ldots+P(B \mid A_n)P(A_n)\]
\end{dinglist}
\pause
\begin{itemize}
\item Such collection of sets $A_1,\ldots,A_n$ is also called a partition of sample space.
\end{itemize}

\end{frame}

%%%%%%%%%%%%%%%%%%%%%%%%%%%%%%%%%%%%%%%%%%
\begin{frame}
\disc{
\footnotesize{Example: a firm is considering a drug-testing program for its employees,
but before it begins it wants to know the scope of the problem if any exists.
Realizing the sensitivity of this issue, the personnel director decides to use a randomized response survey.
It is believed that respondents are more likely to be honest when such forms are used.
Each employee is asked to flip a fair coin. If the coin comes up heads, answer the question
``Do you carpool to work?''.
If the coin comes up tails, answer the question ``Have you used illegal drugs within the last month?''.
Assume that all employees answer the survey honestly. Out of 8000 responses,
1420 answered ``YES''.
Suppose the firm knows that 35\% of its employees carpool to work.
What is the probability that an employee chosen at random used illegal drugs within the last month?
}}
\pause
Let events $A_1 = \{\text{head}\}$, $A_2 = \{\text{tail}\}$, $B = \{ \text{answer YES} \}$.\\
Since $\{A_1, A_2\}$ is a partition, so we can use the law to total probability
\pause
\begin{align*}
P(B)= & P(B \mid A_1)P(A_1)+P(B \mid A_2)P(A_2)\\
P(\text{YES}) = & P(\text{carpool}) P(\text{head}) +P(\text{drug}) P(\text{tail})\\
\frac{1420}{8000} = & 0.35 \times 0.5 +P(\text{drug})\times 0.5\\
P(\text{drug}) &= 0.005
\end{align*}



\end{frame}


%%%%%%%%%%%%%%%%%%%%%%%%%%%%%%%%%%%%%%%%%%
\begin{frame}\frametitle{Recap}

\begin{itemize}
\item Marginal probability: $P(A), P(B)$
\item Joint probability: $P(A \cap B)$
\item Conditional probability
\[ P(A \mid B)=\frac{P(A \cap B)}{P(B)}\]
\pause
\item Multiplication rule:
\[P(A \cap B)=P(A \mid B)\times P(B)\]
%\item Independence: $P(A|B) = P(A)$, $P(A \cap B) = P(A) \times P(B)$
\item Law of total probability: for a partition $\{A_1, A_2, \ldots, A_n\}$ of $S$,
\[P(B)= \sum_{j = 1}^nP(B \mid A_j)P(A_j)\]
\end{itemize}


\end{frame}
%%%%%%%%%%%%%%%%%%%%%%%%%%%%%%%%%%%%%%%%%%


%%%%%%%%%%%%%%%%%%%%%
%\begin{frame}{Outline}
%%\tableofcontents[hideallsubsections,pausections]
%\tableofcontents[hideallsubsections]
%\end{frame}



%%%%%%%%%%%%%%%%%%%%%%%%%%%%%%%%%%%%%%%%%%%
%\begin{frame}%\frametitle{}
%
%\disc{Monty Hall problem: suppose you are on a game show, and you are given the choice of three doors: Behind one door is a car; behind the others, goats. \\
%
%You pick a door, say No.\ 1, and the host, who knows whatÕs behind the doors, opens another door, say No.\ 3, which he knows has a goat.
%He then asks you, ``Do you want to switch to door No.\ 2?'' Is it to your advantage to switch your choice?}
%\pause
%Event $A$: You first pick the correct door.
%Event $B$: You win the car.\\
%\pause
%If you switch, by the law of total probability
%\begin{eqnarray*}
%P(B)&=&P(B \mid A)P(A)+P(B \mid A^c)P(A^c)\\
%&=&0\times 1/3 + 1\times 2/3 = 2/3
%\end{eqnarray*}
%\pause
%If you don't switch, by the law of total probability
%\begin{eqnarray*}
%P(B)&=&P(B \mid A)P(A)+P(B \mid A^c)P(A^c)\\
%&=&1\times 1/3 + 0\times 2/3 = 1/3
%\end{eqnarray*}
%\pause
%Good to switch.
%
%
%\end{frame}

%%%%%%%%%%%%%%%%%%%%%%%%%%%%%%%%%%%%%%%%%%
\section{Bayes theorem}
%%%%%%%%%%%%%%%%%%%%%%%%%%%%%%%%%%%%%%%%%%
\begin{frame}\frametitle{Bayes theorem (also called Bayes rule)}

Suppose events $A_1,\ldots,A_n$  are  disjoint, and
$\bigcup_{i=1}^n A_i = S,$
with $P(A_i) > 0$, $i = 1, 2, \ldots, n$.
Then for any event $B$ with $P(B) > 0$, \pause
\begin{align*}
P(A_i \mid B) = & \frac{P(B \mid A_i)P(A_i)}{\sum\limits_{j = 1}^n P(B \mid A_j)P(A_j)},
\quad i = 1, \ldots, n \\
& = \frac{P(B \mid A_i)P(A_i)}{P(B \mid A_1)P(A_1)+ \cdots + P(B \mid A_n)P(A_n)}
\end{align*}
\pause
\begin{itemize}
\item $P(A_i)$ is often called \hl{prior probability}
\item $P(A_i \mid B)$ is called \hl{posterior probability}.
\end{itemize}

\end{frame}

%%%%%%%%%%%%%%%%%%%%%%%%%%%%%%%%%%%%%%%%%%
\begin{frame}\frametitle{Example: Vassar students}
\begin{itemize}
\item Suppose Vassar has only three possible majors:  math, music,
and econ. And 25\% of the students are in math,
55\% are in music, and the rest are in econ (with no double majors).
\pause
\item Campus folklore say that 90\% of math majors want to go on dates,
compared with 50\% of music and 10\% of econ majors.
\pause
\item Your roommate meets someone at an event and gets a date.
What is the probability that she is dating a math major?
\pause
\item Note that the majors define a finite partition, and the campus
folklore gives us the conditional probabilities $\Pr(B \mid A_i)$.
\pause
\item The point of Bayes' rule is to reverse the conditioning to get
$\Pr(A_i \mid B)$.
\end{itemize}




\end{frame}


%%%%%%%%%%%%%%%%%%%%%%%%%%%%%%%%%%%%%%%%%%
\begin{frame}

\disc{Example: Suppose Vassar has only three possible majors:  math, music,
and econ. And 25\% of the students are in math,
55\% are in music, and the rest are in econ (with no double majors).
Campus folklore say that 90\% of math majors want to go on dates,
compared with 50\% of music and 10\% of econ majors.
Find $P(\text{math} \mid \text{date})$.}
\pause
Events $A_1 = \{ \text{math} \}, A_2 = \{ \text{music} \}, A_3 = \{ \text{econ} \}$ define  a partition. \\
Let $B = \{ \text{date} \}$.
In order to calculate $P(A_1 \mid B)$, use the Bayes theorem:
\pause
\begin{align*}
P(A_1 \mid B)
= & \frac{P(B \mid A_1)P(A_1)}{P(B \mid A_1)P(A_1)+ P(B \mid A_2)P(A_2) + P(B \mid A_3)P(A_3)}\\
= & \frac{0.9 \times 0.25}{0.9 \times 0.25 + 0.5 \times 0.55 + 0.1 \times 0.20 } = 0.43
\end{align*}

\end{frame}
%
%
%%%%%%%%%%%%%%%%%%%%%%%%%%%%%%%%%%%%%%%%%%
%
%\begin{frame}\frametitle{Example: diagnostic tests}
%
%{
%\disc{Lupus is a medical phenomenon where antibodies that are supposed to attack foreign cells to prevent infections instead see plasma proteins as foreign bodies, leading to a high risk of blood clotting. It is believed that 2\% of the population suffer from this disease. \\
%
%The test for lupus is very accurate if the person actually has lupus, however is very inaccurate if the person does not. More specifically, the test is 98\% accurate if a person actually has the disease. The test is 74\% accurate if a person does not have the disease. \\
%
%There is a line from the Fox television show \textit{House}, often used after a patient tests positive for lupus: ``It's never lupus." Do you think there is truth to this statement? }
%}
%
%
%\end{frame}
%
%
%
%%%%%%%%%%%%%%%%%%%%%%%%%%%%%%%%%%%%%%
%
%\begin{frame}
%
%\begin{itemize}
%\item Event $A = \{ \text{lupus} \}, \quad P(A) = 0.02, \quad P(A^c) = 1 - P(A) = 0.98$
%\item Event $B = \{ \text{positive} \}, \quad P(B \mid A) = 0.98$
%\item $P(\text{positive} \mid \text{no lupus}) = P(B \mid A^c) = 1- 0.74 = 0.26$
%\end{itemize}
%
%\pause
%\begin{align*}
%P(\text{lupus} \mid \text{positive}) & = P(A \mid B)  \\
%&= \frac{P(B \mid A)P(A)}{P(B \mid A)P(A)+ P(B \mid A^c)P(A^c)} \\
%&= \frac{0.98\times 0.02}{0.98\times 0.02 + 0.26\times 0.98} \\
%%&= \frac{0.0196}{0.0196 + 0.2548} \\
%&= 0.0714
%\end{align*}
%\pause
%Even when a patient tests positive for lupus, there is only a 7.14\% chance that he/she actually has lupus.
%This is because the background rate of lupus is quite low.
%House may be right.
%\pause
%\ct{\webURL{http://www.ted.com/talks/peter_donnelly_shows_how_stats_fool_juries.html/} From about 11:00 }
%
%
%\end{frame}


%%%%%%%%%%%%%%%%%%%%%%%%%%%%%%%%%%%%%%%%%%

%%%%%%%%%%%%%%%%%%%%%%%%%%%%%%%%%%%%%%%%%%
\begin{frame}\frametitle{Recap}

Bayes theorem
\begin{align*}
P(A_i \mid B) = & \frac{P(B \mid A_i)P(A_i)}{\sum\limits_{j = 1}^n P(B \mid A_j)P(A_j)},
\quad i = 1, \ldots, n
\end{align*}
%\pause
%
%\cl{Suppose in the Monty Hall problem, I tossed a fair coin to decide whether to stay or switch.
%Say I won the car, what's the probability that I switched?}
%
%%\invisible{
%\pause
%\vspace{0.5cm}
%Let $A_1 = \{ \text{switch} \}, A_2 = \{ \text{no switch} \}, B = \{ \text{win} \}$, using Bayes theorem
%\begin{align*}
%P(A_1 \mid B)
%& = \frac{P(B \mid A_1)P(A_1)}
%	{P(B \mid A_1)P(A_1) + P(B \mid A_2)P(A_2)}\\
%& = \frac{\frac{2}{3}\cdot \frac{1}{2}}{\frac{2}{3}\cdot \frac{1}{2} + \frac{1}{3}\cdot \frac{1}{2}}=  \frac{2}{3}
%\end{align*}
%}


\end{frame}


%%%%%%%%%%%%%%%%%%%%%%%%%%%%%%%%%%%%%%%%%%
\section{Independent events}
%%%%%%%%%%%%%%%%%%%%%%%%%%%%%%%%%%%%%%%%%%
\begin{frame}
\frametitle{Product rule for independent events}

\begin{defn}
Events $A$ and $B$ are \hl{independent} if
\[P(A \cap B) = P(A) \times P(B) \]
%\small{Or more generally, $P(A_1~and~\cdots~and~A_k) = P(A_1) \times \cdots \times P(A_k)$}}
\end{defn}

\pause
\begin{itemize}
\item If $A$ and $B$ are independent, then
\[P(A \mid B) = \frac{P(A)P(B)}{P(B)} = P(A).\]
Knowing $B$ doesn't affect the odds of $A$.
\end{itemize}

\end{frame}
%%%%%%%%%%%%%%%%%%%%%%%%%%%%%%%%%%%%%%%%%%
\begin{frame}\frametitle{}
\disc{Example: you toss a coin twice, what is the probability of getting two tails in a row?}
\pause
\begin{itemize}
\item Method 1: sample space $S = \{ HH, TT, HT, TH \}$.\\
Since one out of four possible outcomes match this definition,
the probability is $\frac{1}{4}$.\\
\red{Inefficient way if the number of trials was much higher}.
\pause
\item Method 2: use independence.
\[ P(\text{T on the first toss}) \times  P(\text{T on the second toss}) = \frac{1}{2} \times \frac{1}{2} = \frac{1}{4} \]
\vspace{-0.5cm}
%\item This also verifies the events \{T on the first toss\} and \{T on the second toss\} are independent.
\end{itemize}

\end{frame}

%%%%%%%%%%%%%%%%%%%%%%%%%%%%%%%%%%%%%%%%%%
\begin{frame}\frametitle{Properties of independence}
If $A$ and $B$ are independent, then
\begin{itemize}
\item $B$ and $A$ are independent.
\vspace{1.5cm}

\pause
\item $A$ and $B^c$ are independent. And so are $A^c$ and $B^c$.
\vspace{1.5cm}


\pause
\item Independent is not disjoint.

\vspace{1.5cm}



\end{itemize}

\end{frame}

%%%%%%%%%%%%%%%%%%%%%%%%%%%%%%%%%%%%%%%%%%
\begin{frame}\frametitle{Mutually independent}

\begin{dinglist}{\DingListSymbolA}
\item Three events $A, B, C$ are called \hl{mutually independent} if
\begin{enumerate}
\item \[P(A \cap B) = P(A) P(B)\]
\[P(B \cap C) = P(B) P(C)\]
\[P(A \cap C) = P(A) P(C)\]
\item \[P(A \cap B \cap C) = P(A) P(B) P(C)\]
\end{enumerate}

\pause
\item If only (1) holds but not (2), then $A, B, C$ are called \hl{pairwise independent}.

\end{dinglist}

\end{frame}

%%%%%%%%%%%%%%%%%%%%%%%%%%%%%%%%%%%%%%%%%%
\begin{frame}\frametitle{Example: pairwise indpt but not mutually indpt}
\disc{
Roll two fair 6-sided dice. Set
\begin{itemize}
\item $A = \{ \text{Sum is 7} \}$
\item $B = \{ \text{First roll is 4} \}$
\item $C = \{ \text{Second roll is 3} \}$
\end{itemize}
}

\vspace{-0.5cm}

%\invisible{
\uncover<2->{
\[
\begin{array}{ccccc}
        P(A \cap B) & \stackrel{?}{=} & P(A) &\times & P(B) \\
	\uncover<3->{
	1/36  &			& 1/6 & & 1/6
	}
\end{array}
\invisible{
\hspace{1cm} A, B, C \text{ are pairwise indpt}
}
\]
}

\vspace{-0.7cm}
\uncover<4->{
\[
\begin{array}{ccccc}
        P(B \cap C) & \stackrel{?}{=} & P(B) &\times & P(C) \\
	\uncover<5->{
	1/36  &			& 1/6 & & 1/6
	}
\end{array}
\uncover<8->{
\hspace{1cm} A, B, C \text{ are pairwise indpt}
}
\]
}

\vspace{-0.7cm}
\uncover<6->{
\[
\begin{array}{ccccc}
        P(A \cap C) & \stackrel{?}{=} & P(A) &\times & P(C) \\
	\uncover<7->{
	1/36  &			& 1/6 & & 1/6
	}
\end{array}
\invisible{
\hspace{1cm} A, B, C \text{ are pairwise indpt}
}
\]
}

\vspace{-0.7cm}
\uncover<9->{
\[
\begin{array}{ccccccc}
        P(A \cap B \cap C) & \stackrel{?}{=} & P(A) &\times & P(B) & \times & P(C) \\
	\uncover<10->{
	1/36  &			& 1/6 & & 1/6 & & 1/6
	}
\end{array}
\uncover<11->{
 \text{ NOT mutually indpt}
}
\]

}

%}



\end{frame}

%%%%%%%%%%%%%%%%%%%%%%%%%%%%%%%%%%%%%%%%%%
\begin{frame}%\frametitle{}

\begin{dinglist}{\DingListSymbolA}
\item Events $A_1, A_2, \ldots, A_n$ are called \hl{independent} if
for any $1\leq r \leq n$ of them
 \[P(A_{i_1} \cap A_{i_2} \cap \cdots \cap A_{i_r}) = P(A_{i_1}) P(A_{i_2})  \cdots P(A_{i_r})\]

\pause
\item The key to compute the probability of independent events is to just multiply the probability of the individual events.
\end{dinglist}

\vspace{1cm}
\pause
\disc{A fair coin is tossed 5 times. Compute $P(\text{5 tails})$.}

%\invisible{
\pause
\[ P(T_1\ T_2 \ T_3 \ T_4 \ T_5) = P(T_1)P(T_2)P(T_3)P(T_4)P(T_5) = \frac{1}{2^5}\]
%}
\end{frame}



%%%%%%%%%%%%%%%%%%%%%%%%%%%%%%%%%%%%%%%%%%
%\begin{frame}
%
%\end{frame}
%
%

%%%%%%%%%%%%%%%%%%%%%%%%%%%%%%%%%%%%%%%%%%



%%%%%%%%%%%%%%%%%%%%%%%%%%%%%%%%%%%%%%%%%%
\section{Perceptions of probability and biases}
%%%%%%%%%%%%%%%%%%%%%%%%%%%%%%%%%%%%%%%%%%

\begin{frame}{The Linda problem}
\disc{Linda is 31 years old, single, outspoken, and very bright. She majored in philosophy. As a student she was deeply concerned with issues of discrimination and social justice, and she also participated in anti-nuclear demonstrations. Please rank the following statements by their probability, using 1 for the most probable and 8 for the least probable.
\begin{enumerate}
\item Linda is a teacher in an elementary school.
\item Linda works in a bookstore and takes yoga classes.
\item Linda is active in the feminist movement.
\item Linda is a psychiatric social worker.
\item Linda is a member of the League of Women Voters.
\item Linda is a bank teller.
\item Linda sells insurance.
\item Linda is a bank teller and is active in the feminist movement.
\end{enumerate}
}
\end{frame}


\begin{frame}{The Linda problem cont'd}
\disc{Linda is 31 years old, single, outspoken, and very bright. She majored in philosophy. As a student she was deeply concerned with issues of discrimination and social justice, and she also participated in anti-nuclear demonstrations. Please rank the following statements by their probability, using 1 for the most probable and 8 for the least probable.
\begin{enumerate}
\item Linda is a teacher in an elementary school.
\item Linda works in a bookstore and takes yoga classes.
\item {\color{red}Linda is active in the feminist movement.}
\item Linda is a psychiatric social worker.
\item Linda is a member of the League of Women Voters.
\item {\color{red}Linda is a bank teller.}
\item Linda sells insurance.
\item {\color{red}Linda is a bank teller and is active in the feminist movement.}
\end{enumerate}
}
\end{frame}

\begin{frame}\frametitle{Set theory - intersection}

\begin{itemize}
\item Event $A$ = \{Linda is active in the feminist movement\}
\item Event $B$ = \{Linda is a bank teller\}
\item Event $A \cap B$ = \{Linda is a bank teller and is active in the feminist movement\}.
\pause
\item {\color{red}$P(A \cap B) \leq P(A)$ and $P(A \cap B) \leq P(B)$!}
\item What did our class do? 21 of you responded.
\begin{itemize}
\item Ranking: 1 is the most probable choice, and 8 is the least probable choice
\item Avg(Ranking of event $A$) = 3.38
\item Avg(Ranking of event $B$) = 4.76
\item Avg(Ranking of event $A \cap B$) = 4.62
\item Percentage of people ranking event $A \cap B$ higher than event $B$ $\approx$ 62\% (13 out of 21 responses)
\end{itemize}
\end{itemize}
\end{frame}


\begin{frame}\frametitle{The Linda problem - historical results}
The Linda problem was proposed by Amos Tversky and Daniel Kahneman in 1983.
\begin{itemize}
\item Ranking: 1 is the most probable choice, and 8 is the least probable choice
\item Avg(Ranking of event $A$) = 2.1
\item Avg(Ranking of event $B$) = 6.2
\item Avg(Ranking of event $A \cap B$) = 4.1
\item Percentage of people ranking event $A \cap B$ higher than event $B$ = 90\% (even among those who have had several advanced courses in probability and statistics)
\end{itemize}

\end{frame}


\begin{frame}\frametitle{Representativeness and Conjunction Bias}

\begin{itemize}
\item Representativeness heuristic:
\begin{itemize}
\item people determine the probability of an event by the degree to which the event is similar in essential characteristics to its parent population.
\item it leads people to inflate the believed probability of events that resembles the data available.
\end{itemize}

\pause

\item Conjunction bias:
\begin{itemize}
\item intersecting relatively unrepresentative events (bank teller) with very representative event (feminist).
\item the description of the conjunction of the two (feminist bank teller) is considered more probable than the unrepresentative event (bank teller).
\end{itemize}

\pause

\item Interested in these types of topics? ECON 333 Behavioral Economics!
\end{itemize}


\end{frame}

\end{document}
