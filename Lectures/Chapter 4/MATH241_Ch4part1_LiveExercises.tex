\documentclass[11pt]{article}
\usepackage[top=2.1cm,bottom=2cm,left=2cm,right= 2cm]{geometry}
%\geometry{landscape}                % Activate for for rotated page geometry
\usepackage[parfill]{parskip}    % Activate to begin paragraphs with an empty line rather than an indent
\usepackage{graphicx}
\usepackage{amssymb}
\usepackage{epstopdf}
\usepackage{amsmath}
\usepackage{multirow}
\usepackage{hyperref}
\usepackage{changepage}
\usepackage{lscape}
\usepackage{ulem}
\usepackage{multicol}
\usepackage{dashrule}
\usepackage[usenames,dvipsnames]{color}
\usepackage{enumerate}
\newcommand{\urlwofont}[1]{\urlstyle{same}\url{#1}}
\newcommand{\degree}{\ensuremath{^\circ}}
\newcommand{\hl}[1]{\textbf{\underline{#1}}}



\DeclareGraphicsRule{.tif}{png}{.png}{`convert #1 `dirname #1`/`basename #1 .tif`.png}

\newenvironment{choices}{
\begin{enumerate}[(a)]
}{\end{enumerate}}

%\newcommand{\soln}[1]{\textcolor{MidnightBlue}{\textit{#1}}}	% delete #1 to get rid of solutions for handouts
\newcommand{\soln}[1]{ \vspace{1.35cm} }

%\newcommand{\solnMult}[1]{\textbf{\textcolor{MidnightBlue}{\textit{#1}}}}	% uncomment for solutions
\newcommand{\solnMult}[1]{ #1 }	% uncomment for handouts

%\newcommand{\pts}[1]{ \textbf{{\footnotesize \textcolor{black}{(#1)}}} }	% uncomment for handouts
\newcommand{\pts}[1]{ \textbf{{\footnotesize \textcolor{blue}{(#1)}}} }	% uncomment for handouts

\newcommand{\note}[1]{ \textbf{\textcolor{red}{[#1]}} }	% uncomment for handouts

\begin{document}


\enlargethispage{\baselineskip}

Spring 2021 \hfill Jingchen (Monika) Hu\\

\begin{center}
{\huge MATH 241 Chapter 4 part 1 Live Exercises}	\\
\end{center}
\vspace{0.5cm}

\begin{enumerate}

%%%%%%%%%%%%%%%%%%%%%%%%%%%%%%%%%%%%%%%%%%%%%%

\item Two fair six-sided dice are rolled. Let the random variable $X$ denote the product of the 2 dice. What are possible values of $X$ and their associated probabilities? Just give a few examples and you do not need to calculate the associated probability for all possible values.

\item Two fair six-sided dice are rolled. Let the random variable $X$ denote the product of the 2 dice. Find the probability of: (a) $P(X \leq 2)$, (b) $P(X \leq 35)$.

\item Three fair coins are tossed. Let the random variable $X$ denote the number of heads. Write out the pmf and cdf of $X$.

\item Suppose that the distribution function of $X$ is given by
\begin{align*}
F(b)= 
\begin{cases}
    0,& b < 0\\
    \frac{b}{4},& 0 \leq b < 1 \\
    \frac{1}{2} + \frac{b - 1}{4},& 1 \leq b < 2 \\
    \frac{11}{12}, & 2 \leq b < 3 \\
    1, & 3 \leq b
\end{cases}
\end{align*}
Find $P(X = i), i = 1, 2, 3$.

\item Toss a coin. Suppose the probability of a head is $p$. Let $X$ be
a 0-1 indicator random variable s.t.\
\[
X = \left\{
\begin{array}{ll}
1	& \text{if head is obtained}\\
0	& \text{otherwise}\\
\end{array}
\right.
\]
Compute $\mu = E[X]$.

\item Let the random variable $X$ denote the GP a certain student will earn in this class. Suppose its pmf is
\[ p(0) = 0.05, \quad p(1) = 0.05, \quad p(2) = 0.3, \quad p(3) = 0.4\]
Calculate their expected GP $E[X]$.

\item Let $X$ denote a random variable that takes on any of the values -1, 0, and 2 with respective probability: $P(X = -1) = \frac{1}{5}, P(X = 0) = \frac{1}{5}, P(X = 2) = \frac{3}{5}$. Compute $E[X^3]$.

\item Let $X$ denote a random variable that takes on any of the values -1, 0, and 2 with respective probability: $P(X = -1) = \frac{1}{5}, P(X = 0) = \frac{1}{5}, P(X = 2) = \frac{3}{5}$. Compute: 
\begin{enumerate}
\item $E[2X^2]$
\item $E[4X^2 - 1]$
\end{enumerate}

\end{enumerate}


\end{document} 