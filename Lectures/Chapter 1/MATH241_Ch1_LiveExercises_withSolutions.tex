\documentclass[slidestop,compress,mathserif]{beamer}
%\documentclass[slidestop,compress,mathserif,handout]{beamer}

%\documentclass[xcolor=dvipsnames,handout]{beamer}
%\documentclass[xcolor=dvipsnames]{beamer}

%\documentclass[handout]{beamer}

%%% To get rid of solutions on handouts:
\newcommand{\soln}[1]{\textit{\textcolor{darkGray}{#1}}}				% For slides
%\newcommand{\soln}[1]{ }	% For handouts

% to get pausing to work properly on slides
\newcommand{\hide}[1]{#1}	% For slides
%\newcommand{\hide}[1]{ }	% For handouts


\input{../LectureStyle.tex}



%%%%%%%%%%%%%%%%%%%%%%%%%%%%%%%%%%%%%%%%%%%%%%%%%%%%%%%%%%%%%%%%%%%%%%%%%%%%%%%%%%%%%%%%%%%%%%%

\title[Chapter 1]{Chapter 1}
\subtitle{Combinatorial Analysis}

%%%%%%%%%%%%%%%%%%%%%%%%%%%%%%%%%%%%%%%%%%%%%%%%%%%%%%%%%%%%%%%%%%%%%%%%%%%%%%%%%%%%%%%%%%%%%%%


\author[Jingchen (Monika) Hu]
{Jingchen (Monika) Hu}

\institute[Vassar]
{Vassar College}


\date[MATH 241]
{MATH 241}


\subject{MATH 241}


\begin{document}




%%%%%%%%%%%%%%%%%%%%%

% Title Page

\begin{frame}%[plain]
\titlepage
\end{frame}

%%%%%%%%%%%%%%%%%%%%%
%\addtocounter{framenumber}{-1}
%
%\begin{frame}\frametitle{Annoucement}
%
%\begin{itemize}
%\item HW1: \red{due Thursday, Aug 28th}\\
%Blackboard course page $\longrightarrow$ Content $\longrightarrow$ Homework assignments
%
%\end{itemize}
%
%\end{frame}

%%%%%%%%%%%%%%%%%%%%%

\begin{frame}{Outline}
\tableofcontents[hideallsubsections]
\end{frame}




%%%%%%%%%%%%%%%%%%%%%%%%%%%%%%%%%%%%
\section{The basic rule of counting}
%%%%%%%%%%%%%%%%%%%%%%%%%%%%%%%%%%%%

\begin{frame}\frametitle{The generalized rule of counting}

\begin{dinglist}{\DingListSymbolA}
\item Suppose an experiment consists $r$ different outcomes, with the $i$-th outcome having $n_i$ possibilities,
then together there are
\[
n_1 \times n_2 \times \cdots \times n_r = \prod_{i=1}^r n_i
\]
possibilities for the experiment.
\end{dinglist}

\uncover<2-3>{
\cl{How many different 4-digit pins?
\begin{center}
\begin{tabular}{cccccc}
\underline{\hspace{1cm}} &\underline{\hspace{1cm}} &\underline{\hspace{1cm}} &\underline{\hspace{1cm}}  \\
letter or number& letter or number & letter or number & letter or number
\end{tabular}
\end{center}
}
}

\uncover<3>{
\[
(26 + 10) \times (26 + 10) \times (26 + 10) \times (26 + 10) = 1,679,616
\]
%Try out in R!
}

\end{frame}

%%%%%%%%%%%%%%%%%%%%%%%%%%%%%%%%%%%%
\section{Permutations}
%%%%%%%%%%%%%%%%%%%%%%%%%%%%%%%%%%%%
\begin{frame}[fragile]\frametitle{Permutations}

\disc{
Our example: how many different arrangements of the letters a, b, c?
}

\begin{dinglist}{\DingListSymbolA}
\item Each of these arrangements is a \red{permutation}
\item The order matters!
\item Number of permutations of $n$ different objects
\[
n \times (n-1) \times \cdots 2 \times 1 = n!
\]
\end{dinglist}

\begin{comment}
Factorial in R:
\begin{verbatim}
factorial(3)
\end{verbatim}
\end{comment}

\end{frame}

%%%%%%%%%%%%%%%%%%%%%%%%%%%%%%%%%%%%

%%%%%%%%%%%%%%%%%%%%%%%%%%%%%%%%%%%%
\begin{frame}\frametitle{Permutations of r groups of n objects}
\begin{dinglist}{\DingListSymbolA}
\item Among $n$ objects, if $n_1$ are alike, $n_2$ are alike, \ldots, $n_r$ are alike, then there are
\[
\frac{n!}{n_1! n_2! \cdots n_r!}
\]
different permutations.
\end{dinglist}

\uncover<2-3>{
\cl{Number of permutations of the letters in the word ``Vassar''?}
\begin{enumerate}[(a)]
\item $6! / 2$
\item $5! / 2$
\only<beamer| beamer:2>{\item{ $6!/ 4$}} \only<3>{\item{ \red{$6!/ 4$}}}
\item $6!$
\end{enumerate}
}

\uncover<3>{
%Try out in R!
}

\end{frame}

\begin{frame}\frametitle{Permutation of selecting r items from n objects}
\begin{dinglist}{\DingListSymbolA}
\item If we have $n$ items and want to select $r$ of them,
\[
\#(\text{permutations}) = n \times (n-1) \times \cdots \times (n-r+1) = \frac{n!}{(n-r)!}
\]
\end{dinglist}

\cl{Suppose you have 3 distinctive gifts to give to 8 friends. How many permutations of gift giving strategy do you have?}
\begin{enumerate}[(a)]
\item $3! $
\item $3 ^ 3$
\item $8!$
\solnMult{$8\times 7 \times 6$}
\end{enumerate}

\uncover<2->{
\begin{center}
\begin{tabular}{cccccc}
\underline{\hspace{1cm}} &  &\underline{\hspace{1cm}} & &\underline{\hspace{1cm}}\\
\uncover<3->{
8 & $\times$  & 7 & $\times$ & 6 & $= ~~8! / (8 - 3)!$
}
\end{tabular}
\end{center}
}

\uncover<4->{

%Try out in R!
}
\vspace{1mm}

\uncover<5->{
\underline{What if the order doesn't matter? e.g.\ handshakes.}
}
\end{frame}


%%%%%%%%%%%%%%%%%%%%%%%%%%%%%%%%%%%%
%%%%%%%%%%%%%%%%%%%%%%%%%%%%%%%%%%%%

%%%%%%%%%%%%%%%%%%%%%%%%%%%%%%%%%%%%%%%%%%
\begin{frame}%\frametitle{Additional exercises}

\cl{(a) In how many ways can 4 married couples line up?  \\
(b) What if couples must stand together?}
\pause
Part (a): 8 different people
\[8! = 40320\]
\pause

Part (b): there are $4!=24$ ways that the couples can be arranged, and each
couple can be arranged in $2! = 2$ ways.  So the answer is \pause
\[
(4!) \times (2!) \times (2!) \times (2!) \times (2!) = 384.
\]
\pause
Another way to look at the problem:\\

For the first position, there are 8 choices. For the next position, only one, because the first person's spouse needs to take that position. The next one has 6, next 1 again, etc.. So the answer is

$$ 8 \times 1 \times 6 \times 1 \times 4 \times 1 \times 2 \times 1 = 384.$$
\end{frame}






\begin{frame}\frametitle{Recap}

The basic rule of counting
\begin{itemize}
\item $r$ different outcomes; the $i$-th outcome having $n_i$ possibilities,
then the number of possibilities is
\[
\prod_{i=1}^r n_i
\]

\end{itemize}

Permutations
\begin{itemize}
\item Number of permutations of $n$ different objects is $n!$.
\item Number of permutations of $n$ objects, if $n_1$ are alike, $n_2$ are alike, \ldots, $n_r$ are alike, is
\[
\frac{n!}{n_1! n_2! \cdots n_r!}
\]
\item Number of permutations of selecting $r$ items from $n$ objects
\[
\frac{n!}{(n-r)!}
\]
\end{itemize}

\end{frame}


%%%%%%%%%%%%%%%%%%%%%%%%%%%%%%%%%%%%%%%%%%
\section{Combinations}
%%%%%%%%%%%%%%%%%%%%%%%%%%%%%%%%%%%%%%%%%%
\begin{frame}[fragile]\frametitle{Combinations: order doesn't matter!}

When order matters, there are $r!$ different orderings of the $r$ items selected.

\begin{dinglist}{\DingListSymbolA}
\item If we have $n$ items and want to select $r$ of them,
\[
\#(\text{combinations}) = \frac{n \times (n-1) \times \cdots \times (n-r+1)}{r!} = \frac{n!}{(n-r)!r!}
\]

\pause
\item Define \hl{choose}
\[ {n \choose r} = \frac{n!}{(n-r)!r!}\]
\end{dinglist}

\pause
\begin{itemize}
\item The number ${n \choose r}$ is pronounced as $n$ choose $r$, it is the number of ways to pick
$r$ objects from a set of $n$ distinct objects.

\item $0 \leq r \leq n$, otherwise $0$
\pause
\item $0! = 1$
\end{itemize}

\end{frame}





%%%%%%%%%%%%%%%%%%%%%%%%%%%%%%%%%%%%%%%%%%
\begin{frame}%\frametitle{}
Example: Poker hand. A standard poker deck has 52 cards, in four suits (clubs, diamonds, hearts, spades) of thirteen cards each (2, 3, ..., 10, Jack, Queen, King, Ace).

\cl{How many distinct hands of ``four of a kind" (four of the five cards are of the same rank)?}
\pause
\vspace{0.5cm}
\begin{tabular}{c c c c c}
$ {13 \choose 1}$ & $\times$ & ${4 \choose 4}$ & $\times$ & ${48 \choose 1}$\\
the same rank &&  suits for the same rank && the rest 1 card
\end{tabular}
\end{frame}


\begin{frame}%\frametitle{}
Related to the previous question...
\cl{How many distinct hands of ``four of a suit" (four of the five cards are of the same suit)?}
\pause
\vspace{0.5cm}
\begin{tabular}{c c c c c}
$ {4 \choose 1}$ & $\times$ & ${13 \choose 4}$ & $\times$ & ${39 \choose 1}$\\
the suit &&  4 cards in the same suit && the rest 1 card
\end{tabular}

\pause
\vspace{10mm}
Note that the number of possible ways to get ``four of a kind" ${13 \choose 1} \times {4 \choose 4} \times {48 \choose 1}$ is smaller than the number of possible ways to get ``four of a suit" $ {4 \choose 1} \times {13 \choose 4} \times {39 \choose 1}$, that means it is less possible to get ``four of a kind". In a game, ``four of a kind" wins over ``four of a suit" because of its smaller probability.
\end{frame}


%%%%%%%%%%%%%%%%%%%%%%%%%%%%%%%%%%%%%%%%%%
\begin{frame}
\cl{Among 4 married couples, we want to select a group of 3 people that is not allowed to contain a married couple.
How many choices are there?}
\pause
{\small{
Number of choices if the group can contain married couple(s):
\[
N_1 = {8 \choose 3} = \frac{8!}{3! \times 5!} = 56
\]
\pause
Number of choices if the group contains married couple(s)? \\
Then it can only contain one couple.
\[
N_2 = {4 \choose 1} \times {6 \choose 1} = 24
\]
\pause
The number of choices that the group does not have a couple:
\[
N_1 - N_2 = 32
\]

\pause
Alternatively: there are $8 \times 6 \times 4$ ways of permuting 3 people where no married couple is contained. However, the order plays a role in this calculation, which we do not want. Therefore, there are $\frac{8 \times 6 \times 4}{3!} = 32$ number of choices that the group does not have a couple.

}}
\end{frame}


%%%%%%%%%%%%%%%%%%%%%%%%%%%%%%%%%%%%%%%%%%
\begin{frame}\frametitle{Properties of combinations ${n \choose r} = \frac{n!}{(n-r)!r!}$}

\begin{enumerate}
\item
\[{n \choose 1} = n ~~~~~~~~~~~~~~~~~~~~~~~~ {n \choose n} = 1~~~~~~~~~~~~ ~~~~~~~~~~~~\]
\pause
\item
\[{n \choose r} = {n \choose n - r} ~~~~~~~~~~~~~~~~~~~~~~~~~~~~~~~~~~~~~~~~~~~~~~~~~\]
\pause
\item
\[
{n \choose r} = {n-1 \choose r} + {n-1 \choose r-1}, \quad 1\leq r \leq n ~~~~~~~~~~~~~~~~~
\]
\end{enumerate}
\end{frame}

%%%%%%%%%%%%%%%%%%%%%%%%%%%%%%%%%%%%%%%%%%
\begin{frame}\frametitle{Binomial theorem}

\[
(a + b)^n = \sum_{k = 0}^n {n \choose k} a^k b^{n-k}
\]
{\it Proof: 1) mathematical induction or 2) combinatorial consideration.}\\
\pause
Think about license plates that are formed by $n$ digits, where each digit can be a letter or a number.
$a = 26, b = 10$. \\
\pause
Total number of distinct plates:
\[
N = (a + b)^n = N_0 + N_1 + \cdots + N_n,
\]
where $N_k$ is the number of distinct plates that contains exactly $k$ number of letters.
\pause
\[
N_k = {n \choose k} \times a^k \times b^{n-k}
\]
\vfill
%\Note{By definition, $0! = 1$}
\end{frame}



%%%%%%%%%%%%%%%%%%%%%%%%%%%%%%%%%%%%%%%%%%
\begin{frame}
\cl{How many subsets are there of the set $\{1, 2, \ldots, n\}$?}
%\begin{enumerate}[(a)]
%\solnMult{$2^n$}
%\item ${n \choose 2}$
%\item $\sum_{k = 1}^n {n \choose k}$
%\item $n!$
%\end{enumerate}

\pause
For each $0 \leq k \leq n$, there are ${n \choose k}$ different subsets of size $k$. Then
\[
\#(\text{subsets}) = \sum_{k = 0}^n {n \choose k} %= \sum_{k = 0}^n {n \choose k} 1^k 1^{n-k} = (1 + 1)^n
\]
\pause
Use the Binomial Theorem to simplify
\[
\sum_{k = 0}^n {n \choose k} = \sum_{k = 0}^n {n \choose k} 1^k 1^{n-k} = (1 + 1)^n = 2^n
\]
\pause
\underline{Any other way to solve this question?}

\pause
Each element can be either in the subset or out of the subset (2 choices); there are $n$ elements; therefore the number of subsets is $2^n$ (basic rule of counting).
\end{frame}

%%%%%%%%%%%%%%%%%%%%%%%%%%%%%%%%%%%%%%%%%%
\section{Multinomial coefficients}
%%%%%%%%%%%%%%%%%%%%%%%%%%%%%%%%%%%%%%%%%%
\begin{frame}\frametitle{Multinomial coefficients}

\begin{dinglist}{\DingListSymbolA}
\item  \hl{Multinomial coefficient}: a set of $n$ distinct items is to be
divided into $r$ distinct groups of respective sizes $n_1, \ldots, n_r$, where $n_1 + n_2 + \cdots + n_r = n$.
Number of possible divisions is
\[
{n \choose n_1, n_2, \ldots, n_r} \stackrel{\text{def}}{=} \frac{n!}{n_1!n_2! \cdots n_r!}
\]
\end{dinglist}

\end{frame}

%%%%%%%%%%%%%%%%%%%%%%%%%%%%%%%%%%%%%%%%%%
\begin{frame}%\frametitle{Multinomial Theorem}

\begin{itemize}
\item
\[
{n \choose n_1, n_2, \ldots, n_r} = {n \choose n_1}{n-n_1 \choose n_2}{n-n_1-n_2 \choose n_3}\cdots{n_r \choose n_r}
\]
\pause

\item When $r = 2$, becomes binomial coefficient (choose function)
\[
{n \choose n_1, n_2} = {n \choose n_1}
\]
Note that $n_1 + n_2 = n$

\pause
\item Multinomial Theorem
\[
(a_1 + a_2 + \cdots + a_r)^n = \sum_{n_1 + \cdots + n_r = n} {n \choose n_1, n_2, \ldots, n_r} a_1^{n_1} a_2^{n_2} \cdots a_r^{n_r}
\]

\pause
\item The Binomial theorem is a special case when $r = 2$.
\end{itemize}

\end{frame}

%%%%%%%%%%%%%%%%%%%%%%%%%%%%%%%%%%%%%%%%%%


%%%%%%%%%%%%%%%%%%%%%%%%%%%%%%%%%%%%
\section{Recap}
%%%%%%%%%%%%%%%%%%%%%%%%%%%%%%%%%%%%

\begin{frame}\frametitle{Recap}

\begin{itemize}

\item From $n$ distinct items, number of ways to draw $r$ of them
\begin{center}
\begin{tabular}{c|cc}
					& without replacement 			& with replacement \\
\hline				
order matters			& $n! / (n-r)!$					& $n^r$ \\
&&\\
order doesn't matter		& ${n \choose r} = n! / (n-r)!r!$		&  see Ch1.6$^*$\\
&&\\
\end{tabular}
\end{center}


\end{itemize}

%\Note{Ch1.6$^*$: students enrolled in 6000 are encouraged to know.}
\end{frame}

\begin{frame}\frametitle{Recap}
Order matters or not?
\pause
\begin{itemize}

\item Textbook Example 5b: Ten children are to be divided into an $A$ team and a $B$ team of 5 each. The $A$ team will play in one league and the $B$ team in another. How many different divisions are possible?
\pause
$${10 \choose 5} = \frac{10!}{5! \times 5!} = 252.$$
\pause

\item Textbook Example 5c: In order to play a game of basketball, 10 children at a playground divide themselves into two teams of 5 each. How many different divisions are possible?
\pause
$$\frac{{10 \choose 5}}{2!} = \frac{10!}{5! \times 5! \times 2!} = 126.$$

\end{itemize}
\end{frame}

\begin{frame}\frametitle{Recap}
\begin{itemize}
\item Binomial theorem
\[
(a + b)^n = \sum_{k = 0}^n {n \choose k} a^k b^{n-k}
\]
License plate problem?

\pause
\item Multinomial coefficient
\[
{n \choose n_1, n_2, \ldots, n_r} = \frac{n!}{n_1!n_2! \cdots n_r!}
\]

Connection?

$$1, 1, \cdots, 1, 2, 2, \cdots 2, \cdots, r, r, \cdots r$$
($n_1$ of $1$, $n_2$ of $2$, ..., $n_r$ of $r$.)\\
\pause
\vspace{1mm}
{\color{red} Each permutation yields a division of the items $\rightarrow$ multinomial. }

\end{itemize}
\end{frame}
%%%%%%%%%%%%%%%%%%%%%%%%%%%%%%%%%%%%%%%%%%
\begin{frame}
\cl{In a well shuffled deck of 52 cards, how many ways can the 4 Aces to be together?}
%\begin{enumerate}[(a)]
%\item $4! \times 48!$
%\solnMult{$4! \times 49!$}
%\item ${52 \choose 4}$
%\item ${52 \choose 4} \times {52 \choose 48} $
%\end{enumerate}

\pause
Treat the 4 Aces together as one object, then the total number of objects are $49$.\\
Also consider the number of permutations among the four aces: $4!$

\pause
\[N = 49! \times 4!\]

\end{frame}


%%%%%%%%%%%%%%%%%%%%%%
%\begin{frame}{Outline}
%%\tableofcontents[hideallsubsections,pausections]
%\tableofcontents[hideallsubsections]
%\end{frame}



\end{document}
